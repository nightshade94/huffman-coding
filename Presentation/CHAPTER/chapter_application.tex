\section{Discover the Application of Algorithm}
Huffman coding is a fundamental of lossless data compression so it is applied in various fields such as File Compression, Multimedia Compression, Data Transmission and Networking, Artificial Intelligence and Machine Learning, Hardware and Embedded Systems.
\begin{enumerate}[label=\textbf{\Alph*.}]
    \item \textbf{File Compression}
    \begin{itemize}
    \item Huffman coding is a common algorithm used in file compression because it helps to reduce file size without losing information.
    \item For example:
        \begin{itemize}
            \item 7-Zip: a popular open-source program for archiving files. It uses Deflate compression which is a combination of LZ77 and Huffman coding.
            \item WinZip and WinRAR: These are commercial file archiving tools that also use DEFLATE compression, including Huffman coding.
            \item macOS Archive Utility (built-in .zip functionality): The built-in archive utility in macOS creates and extracts ZIP files and it relies on DEFLATE.
            \item Command-line tools (gzip, zip): common tools found in Unix-like systems (Linux, macOS), their primary compression method is Deflate.
        \end{itemize}
    \end{itemize}
    \item \textbf{Multimedia Compression}
    \begin{itemize}
    \item Huffman coding is also used in compressing images, audio, and video. The \\algorithm can reduce file sizes while maintaining quality.
    \item For example:
        \begin{itemize}
            \item JPEG (Image Compression): It uses Huffman coding to compress pixel data after Discrete Cosine Transform (DCT).
            \item MP3 (Audio Compression): Huffman coding is used to compress quantized frequency data. It helps to reduce redundancy in audio streams.
            \item MPEG-4 (Video Compression): Huffman coding is used to compress motion vector data in video frames.
        \end{itemize}
    \end{itemize}
    \item \textbf{Data Transmission and Networking}
    \begin{itemize}
    \item By reducing the amount of data sent over the internet, Huffman coding helps to improve network efficiency.
    \item For example:
        \begin{itemize}
            \item HTTP Content-Encoding (GZIP): Web servers use GZIP which includes Huffman coding to compress CSS, HTML, and Javascript files and speed up web loading times.
            \item Fax Machines and Telephony: They use Group 3 fax encoding which includes Huffman coding to compress text-based data for faster transmission. 
        \end{itemize}
    \end{itemize}
    \item \textbf{Artificial Intelligence and Machine Learning}
    \begin{itemize}
    \item Huffman coding is applied in these fields for efficient memory usage.
    \item For example:
        \begin{itemize}
            \item Decision Tree Compression: optimize large decision trees in machine learning.
            \item Natural Language Processing (NLP): Huffman coding is applied in dictionary-based text compression for language modeling.
        \end{itemize}
    \end{itemize}
    \item \textbf{Hardware and Embedded Systems}
    \begin{itemize}
    \item To compress real-time data, hardware chips, and embedded devices need to implement Huffman coding.
    \item For example:
        \begin{itemize}
            \item GPUs and FPGAs: real-time compression in high-performance computing (HPC).
            \item Data Storage Systems: reduce file size in SSDs and HDDs.
        \end{itemize}
    \end{itemize}
\end{enumerate}