\section{Compare with other Compression Algorithms}
\subsection{Introduce}
Huffman Coding is a solid foundation for data compression. While being fundamental, its performance and suitability must be evaluated against other lossless compression algorithms to have a clear picture of the world of compression algorithms. This section provides a comprehensive comparison between Huffman Coding and other significant lossless compression algorithms, especially RLE (Run-Length Encoding), Arithmetic Coding, and the Lempel-Ziv (LZ) family of methods. This evaluation will primarily focus on key performance metrics, including compression efficiency, computational complexity (encoding/decoding speeds), adaptability to varying data characteristics, and associated overhead.\newline
Let's assume that \(N\) represents the length of the input data and \(M\) speaks for the size of the character set.
\subsection{Huffman Coding vs RLE}
\begin{table}[H]
    \centering
    \renewcommand{\arraystretch}{1.5}
    \setlength{\tabcolsep}{8pt} % Giảm khoảng cách giữa các cột để tiết kiệm không gian
    \rowcolors{2}{gray!15}{white}
    \begin{tabular}{|p{0.15\linewidth}|p{0.35\linewidth}|p{0.35\linewidth}|}
        \hline
        \rowcolor{gray!30}
        \textbf{Feature} & \textbf{Huffman Coding} & \textbf{Run-Length Encoding(RLE)}\\
        \hline
        Principle & Variable-length codes based on symbol frequencies. & Replaces repeated substrings with references.\\\hline
        Complexity & Time:   O(N + M log M). \newline Space: O(M). &Time:   O(N). \newline Space: O(N).\\\hline
        Adaptivity & Static version is not adaptive; adaptive versions exist but are more complex. & Not inherently adaptive.\\\hline
        Overhead & Requires storing the Huffman tree (or frequencies) to decode. & Minimal overhead (just the counts).\\\hline
        Use Cases & General-purpose compression, JPEG, MP3, GZIP. & Simple images (e.g., fax), data with long runs.\\\hline
        Strengths & Good overall compression, widely supported. & Very fast and simple for suitable data.\\\hline
        Weaknesses & Can’t take advantage of correlations betweens symbols. & Poor compression for data without long runs.\\\hline
    \end{tabular}
    \caption{Huffman Coding vs RLE}
    \label{tab:Huffman_vs_RLE}
\end{table}
\subsection{Huffman Coding vs Arithmetic Coding}  
\begin{table}[H]  
    \centering  
    \renewcommand{\arraystretch}{1.5} % Giảm khoảng cách giữa các dòng  
    \setlength{\tabcolsep}{8pt} % Giảm khoảng cách giữa các cột  
    \rowcolors{2}{gray!15}{white}   
    \begin{tabular}{|p{0.15\linewidth}|p{0.35\linewidth}|p{0.35\linewidth}|}  
    \hline  
    \rowcolor{gray!30}  
        \textbf{Feature} & \textbf{Huffman Coding} & \textbf{Arithmetic Coding}\\\hline  
        Principle & Variable-length codes based on symbol frequencies. & Encodes the entire message as a single fraction.\\\hline  
        Complexity & Time:   O(N + M log M). \newline Space: O(M). & Time: O(N + M). \newline Space: O(M).\\\hline  
        Adaptivity & Static version is not adaptive; adaptive versions exist but are more complex. & Can be easily made adaptive. \\\hline  
        Overhead & Requires storing the Huffman tree (or frequencies) to decode. & Minimal overhead (typically just model parameters).\\\hline  
        Use Cases & General-purpose compression, JPEG, MP3, GZIP. & JPEG 2000, JBIG2, high-performance compression.\\\hline  
        Strengths & Good overall compression, widely supported. & Better compression, especially for small alphabets.\\\hline  
        Weaknesses & Can’t take advantage of correlations between symbols. & More complex, slower encoding/decoding.\\\hline  
    \end{tabular}  
    \caption{Huffman Coding vs Arithmetic Coding}  
    \label{tab:Huffman_vs_Arithmetic}  
\end{table}  

\subsection{Huffman Coding vs Lempel-Ziv Algorithms (LZ77, LZ78, LZW)}  
\begin{table}[H]  
    \centering   
    \renewcommand{\arraystretch}{1.5} % Giảm khoảng cách giữa các dòng  
    \setlength{\tabcolsep}{8pt} % Giảm khoảng cách giữa các cột  
    \rowcolors{2}{gray!15}{white}  
    \begin{tabular}{|p{0.15\linewidth}|p{0.35\linewidth}|p{0.35\linewidth}|}  
    \hline  
    \rowcolor{gray!30}  
        \textbf{Feature} & \textbf{Huffman Coding} & \textbf{Lempel-Ziv Algorithms}\\\hline  
        Principle & Variable-length codes based on symbol frequencies. & Replaces repeated substrings with references. \\\hline  
        Complexity & Time:   O(N + M log M). \newline Space: O(M). & Time: O(N). \newline Space: O(N).\\\hline  
        Adaptivity & Static version is not adaptive; adaptive versions exist but are more complex. & Adaptive, learns patterns in the data. \\\hline  
        Overhead & Requires storing the Huffman tree (or frequencies) to decode. & Minimal overhead (dictionary is built dynamically).\\\hline  
        Use Cases & General-purpose compression, JPEG, MP3, GZIP. & GZIP, ZIP, PNG (LZ77 + Huffman), GIF (LZW).\\\hline  
        Strengths & Good overall compression, widely supported. & Good for data with repeating patterns, adaptive.\\\hline  
        Weaknesses & Can’t take advantage of correlations between symbols. & Can be slower than Huffman.\\\hline  
    \end{tabular}  
    \caption{Huffman Coding vs Lempel-Ziv Algorithms}  
    \label{tab:Huffman_vs_Lempel-Ziv}  
\end{table}  
\subsection{Huffman Coding vs Burrows-Wheeler Transform (BWT)}  
\begin{table}[H]  
    \centering  
    \renewcommand{\arraystretch}{1.5} % Giảm khoảng cách giữa các dòng  
    \setlength{\tabcolsep}{8pt} % Giảm khoảng cách giữa các cột  
    \rowcolors{2}{gray!15}{white}   
    \begin{tabular}{|p{0.15\linewidth}|p{0.35\linewidth}|p{0.35\linewidth}|}  
    \hline  
        \rowcolor{gray!30}  
        \textbf{Feature} & \textbf{Huffman Coding} & \textbf{Burrows-Wheeler Transform}\\\hline  
        Principle & Variable-length codes based on symbol frequencies. & Reorders data to group similar characters together. Not a compression algorithm itself, but a preprocessor.\\\hline  
        Complexity & Time: O(N + M log M). \newline Space: O(M). & Time: O(N log N) \newline Space: O(N).\\\hline  
        Adaptivity & Static version is not adaptive; adaptive versions exist but are more complex. & Not applicable. \\\hline  
        Overhead & Requires storing the Huffman tree (or frequencies) to decode. & Minimal overhead, needs to store the index of the original string.\\\hline  
        Use Cases & General-purpose compression, JPEG, MP3, GZIP. & bzip2\\\hline  
        Strengths & Good overall compression, widely supported. & Very high compression ratio, especially for text.\\\hline  
        Weaknesses & Can’t take advantage of correlations between symbols. & Relatively slow.\\\hline  
    \end{tabular}  
    \caption{Huffman Coding vs Burrows-Wheeler Transform}  
    \label{tab:Huffman_vs_BWT}  
\end{table}  
\subsection{Conclusion}
There are no perfect compression algorithms, the performance of these compression algorithms is heavily dependent on the input data and specific requirements (compression ratio, speed, memory usage, etc.). Huffman Coding is a versatile and widely used technique, often used with other algorithms to create a more complex compression scheme (JPEG, GZIP, DEFLATE).